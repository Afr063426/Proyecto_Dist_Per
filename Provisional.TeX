con alguna distribuci\'on y una Pareto generalizada para la cola, las ventajas de esto es que 
se obtiene el umbral y no se desperdician datos. Tambi\'en han surgido de la teor\'ia de valores
extremos otras teor\'ias en las que se contempla la frecuencia y la severidad en lo que llaman 
distribuciones compuestas de valores extremos. 
\par Como se puede observar hasta el momento se ha discutido el hecho de que se toman los valores
sobre el umbral que ser\'a el hiperpar\'ametro del modelo, pero no se ha mencionado como es seleccionado este. Hasta el momento tal y como 
lo proponen Klugman et al (2013) y  Ougutu y Rono (2021) la selecci\'on de dicho par\'ametro se
realiza en la mayor\'ia de los casos mediante gr\'aficos, lo que vuelve subjetivo la selecci\'on 
de este par\'ametro. Entre los m\'etodos se eneucntran el gr\'afico de vida residual media, gr\'afico
de estabilidad del par\'ametro y gr\'afico de Gertensgarbe y Werner. No se debe descartar lo \'util 
que pueden ser los gr\'aficos, sin embargo, seg\'un el hiperpar\'ametro seleccionado as\'i ser\'a el
grado de ajuste del modelo y es un intercambio entre el sesgo y la varianza. Si el par\'ametro 
es bajo entonces se viola el argumento que permite emplear una Pareto generalizada y si el umbral es 
demasido alto entonces se tendr\'an pocas observaciones lo que generar\'a una alta varianza.


%-------------------------------------------------------------------------------------------
\par Como se ya se mencion\'o anteriormente existen distintos m\'etodos para la selecci\'on de los
hiperpar\'ametros entre ellos se encuentran seg\'un \textcolor{red}{Ogutu y Rono (2020)} 
el gr\'afico de vida media residual, gr\'afico de estabilidad del par\'ametro, y gr\'afico
de Gertensgarbe y Werner. En el primer m\'etodo se propone la construcci\'on de un gr\'afico
en el cual se busca encontar un punto a partir del cual se tiene un comportamiento lineal,
en el segundo por otro consiste en encontrar un punto a partir del cual se encuentra
un cierto grado de estabilidad y finalmente el \'ultimos se basa en la construcci\'on de dos
series de las cuales se obtiene una sucesi\'on de puntos y el hiper\'ametro es la intersecci\'on
de los puntos. Sin embargo, tal y como lo proponen \textcolor{red}{Andreev (2012)} existen otros 
m\'etodos como el gr\'afico de Hill. 

\par Tambi\'en se han propuesto reglas emp\'iricas. Autores como \textcolor{red}{DuMouchel (1983)} que propone se debe tomar el quantil
0.9, justificando esta selecci\'on como un compromiso entre la necesidad pr\'actica de estimar
los par\'ametros y el deseo te\'orico de describir el comportamiento de la distribuci\'on.

\par Todos estos estimadores tienen una ventaja que es el poco consumo computacional que 
tienen. Sin embargo, \text{red}{Nakamura (2021)} es cr\'itico de estos m\'etodos al destacar
que estos m\'etodos son subjetivos, afectan directamente la estimaci\'on de los par\'ametros y
no son autom\'aticos, siempre requieren de un criterio experto.
Este mismo autor destaca casi inexistencia de m\'etodos que no sean del tipo emp\'irico o gr\'afico.
Esto se puede deber a que anteriormente no se contaba con la capacidad computacional con la que
se cuenta actualmente. \textcolor{red}{Nakamura (2021)} propone en su trabajo un m\'etodo mediante
el cual se busca encontrar la estabilidad de los par\'ametros mediante la estimaci\'on MLE de los
par\'ametros hasta encontrar aquellos en los que se alcance una estabilidad.
Lo que sucede con este m\'etodo se tiene un alto costo computacional. Sin embargo, el autor
propone solucionar esto haciendo uso algoritmos de Machine Learning. En donde en este trabajo
se entender\'a Machine Learning de acuerdo expuesto por \textcolor{red}{Ławrynowicz y Tresp (2014)},
\textcolor{red}{Wehle (2017)}, Shalev-Schwartz y Ben-David (2014), como el uso de
distintos algoritmos o métodos que tienen como objetivo desarrollar programas computacionales, 
que son capaces de encontrar patrones en conjuntos de datos y de cierta forma aprender de estos.

\par El algoritmo de Machine Learning empleado por \textcolor{red}{Nakamura (2021)} es el de optimizaci\'on
bayesiana. Que como lo propone \textcolor{Frazier (2018)} es un algoritmo que permite la optimizaci\'on 
de funciones las cuales pueden tomar un largo periodo de tiempo en ser evaluadas (20 minutos o una hora).
Este algoritmo se ha vuelto popular en los \'ultimos a\~nos para estimar los hiperpar\'ametros en Machine Learning.

\par Otro m\'etodo que ha sido propuesto para la selecci\'on del hiperpar\'ametro de
forma autom\'atica ha sido el propuesto por \textcolor{red}{Bader et al. (2018)}. Donde en
dicho algoritmo se emplean distintos ajustes de bondad de ajuste como son el de Anderson-Darling,
y Cram\'er-Von Mises, test de Moran y score de Rao, y estudio de poder. El objetivo de este algoritmo
es automatizar la escogencia del umbral de ajust y eliminar la subjetividad que se tiene mediante
la selecci\'on empleando gr\'aficos.