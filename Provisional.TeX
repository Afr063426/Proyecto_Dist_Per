con alguna distribuci\'on y una Pareto generalizada para la cola, las ventajas de esto es que 
se obtiene el umbral y no se desperdician datos. Tambi\'en han surgido de la teor\'ia de valores
extremos otras teor\'ias en las que se contempla la frecuencia y la severidad en lo que llaman 
distribuciones compuestas de valores extremos. 
\par Como se puede observar hasta el momento se ha discutido el hecho de que se toman los valores
sobre el umbral que ser\'a el hiperpar\'ametro del modelo, pero no se ha mencionado como es seleccionado este. Hasta el momento tal y como 
lo proponen Klugman et al (2013) y  Ougutu y Rono (2021) la selecci\'on de dicho par\'ametro se
realiza en la mayor\'ia de los casos mediante gr\'aficos, lo que vuelve subjetivo la selecci\'on 
de este par\'ametro. Entre los m\'etodos se eneucntran el gr\'afico de vida residual media, gr\'afico
de estabilidad del par\'ametro y gr\'afico de Gertensgarbe y Werner. No se debe descartar lo \'util 
que pueden ser los gr\'aficos, sin embargo, seg\'un el hiperpar\'ametro seleccionado as\'i ser\'a el
grado de ajuste del modelo y es un intercambio entre el sesgo y la varianza. Si el par\'ametro 
es bajo entonces se viola el argumento que permite emplear una Pareto generalizada y si el umbral es 
demasido alto entonces se tendr\'an pocas observaciones lo que generar\'a una alta varianza.